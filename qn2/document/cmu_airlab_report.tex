\documentclass{article}
\usepackage[a4paper, tmargin=1in, bmargin=1in]{geometry}
\usepackage[utf8]{inputenc}
\usepackage{graphicx}
\usepackage{parskip}
\usepackage{pdflscape}
\usepackage{listings}
\usepackage{hyperref}
\usepackage{gensymb}
\usepackage{amsmath}

\newcommand\Problem[1]{
  \\
  \textbf{Q#1.}
}
\newcommand\Sol[1]{
  \\
  \textbf{A#1.}
  \\
}
\newcommand\tab[1][0.5cm]{\hspace*{#1}}

\title{CMU Airlab Problem Statement}
\author
{
  Arka Sadhu - 140070011\\
  3rd Year Btech. Electrical Engineering
}
\date{December 2016}

\begin{document}
\maketitle

\section*{Problem Statement}
\Problem{1} Name one advantage and one disadvantage of using Euler angles (e.g. Roll-Pitch-Yaw), unit quaternions, rotation matrices and axis/angle representations of rotations.
\Sol{1}
\begin{itemize}
\item Euler Angles:
  \begin{itemize}
  \item Advantages:

    \begin{itemize}
    \item Only 3 parameters need to be stored.
    \item It is more human understandable and good for decomposing rotations into individual degrees of freedom.
    \end{itemize}

  \item Disadvantages:
    \begin{itemize}
    \item Interpolation is difficult.
    \item There are ambiguities in the order of Matrix Rotations.
    \item If the middle rotation is by 90$\degree$, it results in a Gimbal Lock.
    \end{itemize}
  \end{itemize}
  
\item Unit Quaternions:
  \begin{itemize}
  \item Advantages:
    \begin{itemize}
    \item There is no ambiguity, no gimbal lock, and interpolation can be done using slerp, producing smooth rotations.
    \item It requires only 4 parameters to be stored.
    \item Even translations can be accomodated using Dual Quaternions.
    \item Computation is efficient for computers.
    \item Consecutive rotation is simply multiplication of two quternions.
    \end{itemize}

  \item Disadvantages:
    \begin{itemize}
    \item In general it is not easy to visualize Quaternions, and need to be converted to Axis Angle Representation.
    \item The Quaternion-Algebra is quite involved.
    \item There exists redundancy, ie two quaternions may imply the same rotation.
    \end{itemize}
  \end{itemize}

\item Rotation Matrices
  \begin{itemize}
  \item Advantages:
    \begin{itemize}
    \item Rotations can be easily concatenated.
    \item They directly give the new axis of rotations, which may be required in some applications.
    \item With homogenous coordinates it is very easy to incorporate Translation too.
    \end{itemize}

  \item Disadvantages:
    \begin{itemize}
    \item Rotation Matrices need 9 parameters to be stored.
    \item The matrices must be orthogonal, and often due to floating point inaccuracy, they do not remain orthogonal, and the elements of the matrix may need to be recomputed.
    \end{itemize}
  \end{itemize}

\item Axis Angle Representation
  \begin{itemize}
  \item Advantages:
    \begin{itemize}
    \item Very easy to visualize.
    \item Can be very easily converted to quaternions.
    \end{itemize}
  \item Disadvantages:
    \begin{itemize}
    \item If $\theta = 0$ then axis is arbitrary.
    \item For $\theta$ and $\theta + 2*k*\pi$ produce the same result.
    \end{itemize}
  \end{itemize}
\end{itemize}

\Problem{2} Let $R^a = (\theta_{roll}^a, \theta_{pitch}^a, \theta_{yaw}^a)$ and $R^b = (\theta_{roll}^b, \theta_{pitch}^b, \theta_{yaw}^b)$ be a rotation corresponding to the following Roll-Pitch-Yaw angles (ZYX conventions):
$$\theta_{roll}^a = \pi/4 \hspace{2.5cm} \theta_{roll}^b = -\pi/3 $$
$$\theta_{pitch}^a = 0 \hspace{2.5cm} \theta_{roll}^b = 0 $$
$$\theta_{yaw}^a = \pi/3 \hspace{2.5cm} \theta_{roll}^b = 0 $$
Compute 3x3 rotation matrices corresponding to $R^a$ and $R^b$. Then compare $R^aR^b$ and $R^bR^a$ and give physical meaning to these two.
\Sol{2}
\[
  R^a =
  \begin{bmatrix}
    0.707107 & -0.353553 & 0.612372\\
 0.707107 & 0.353553 & -0.612372\\
        0 & 0.866025  &     0.5\\
  \end{bmatrix}
\]
\\
\[
  R^b =
  \begin{bmatrix}
          0.5  & 0.866025    &     0\\
-0.866025   &    0.5     &   -0\\
       -0    &     0     &    1\\
  \end{bmatrix}
\]
\\
\[
  R^aR^b =
  \begin{bmatrix}
    0.65974 & 0.435596 & 0.612372\\
    0.0473672 & 0.789149& -0.612372\\
    -0.75 & 0.433013   &    0.5\\
  \end{bmatrix}
\]
\\
\[
  R^bR^a =
  \begin{bmatrix}
    0.965926 &  0.12941 &-0.224144\\
    -0.258819&  0.482963& -0.836516\\
    0&  0.866025&       0.5\\
  \end{bmatrix}
\]
\\

We note that even though matrices are not commutative in general, they are associative. Let $R^{ab} = R^aR^b$. Clearly $R^{ab}$ is inturn a Rotation matrix. When $R^{ab}$ is applied to some other matrix,i.e. pre-multiplied to a matrix (say $R^{ab}*M$) the multiplication can be viewed in two steps, first multiplication with $R^b$ and then with $R^a$. Similarly with $R^{ba}$, the transformation can be viewed as successive rotations.

Since rotations in 3D are in general not commutative, here too, $R^{ab} \neq R^{ba}$.
\\
\Problem{3} Compute the quaternions $q^a$ and $q^b$ equivalent to the matrices $R^a$ and $R^b$ . Are these quaternions a unique representation of these rotations?
\Sol{3}
Representing Quaternions in the form of [w x y z]
\[
  q^a =
  \begin{bmatrix}
    0.800103 & 0.46194 & 0.191342 & 0.331414\\
  \end{bmatrix}
\]
\\
\[
  q^b =
  \begin{bmatrix}
    0.866025 &
       0 &
       0 &
    -0.5 \\

  \end{bmatrix}
\]
\\
For Quaternions we note that $q$ and $-q$ both give the same rotation. Hence these Quaternions are not a unique representation of these rotations. Other than this the relationship is unique.

\Problem{4} Compute two compositions of the rotations using quaternions, $q^c = q^aq^b$ and 
$q^d = q^bq^a$ . Are $q^c$ and $q^d$ the same? Compute the relative rotation between $q^a$ and
$q^b$ as $q^e = q^a(q^b)^{-1}$ . Then compute the composition $q^f = q^eq^b$ . Verify that $q^f$ and
$q^a$ are the same.
\Sol{4}
\[
  q^c =
  \begin{bmatrix}
    0.858616&
    0.304381&
    0.396677&
    -0.113039\\
  \end{bmatrix}
\]
\\
\[
  q^d =
  \begin{bmatrix}
    0.858616&
    0.495722&
    -0.0652631&
    -0.113039\\
  \end{bmatrix}
\]
\\
As expected $q^c$ and $q^d$ are not the same, since rotations in 3d are in general not commutative.
\[
  q^e =
  \begin{bmatrix}
    0.527203&
    0.495722&
    -0.0652631&
    0.687064\\
  \end{bmatrix}
\]
\\
\[
  q^f =
  \begin{bmatrix}
    0.800103&
    0.46194&
    0.191342&
    0.331414\\
  \end{bmatrix}
\]
\\
As expected $q^f = q^a$.

\Problem{5} How does axis/angle representation relate to quaternion? For example, how to
convert between quaternion q = (x, y, z, w) and angle/axis (n, θ)? Is the conversion
unique or not? If not, how many?
\Sol{5}
Considering the representation of the quaternion as $q = [v \tab s]$, where v is the complex part
and s is the scalar part. Therefore $v = [x \tab y \tab z]$ and $s = w$.

The axis angle and quaternions are closely related. Assumption is that the quaternion is a unit quaternion.
\begin{itemize}
\item Quaternion $[x y z w]$ to Angle Axis $(n,\theta)$:
  $$\theta = 2 * \cos^{-1} w$$
  $$n_x = \frac{x}{\sqrt{1 - w^2}}$$
  $$n_y = \frac{y}{\sqrt{1 - w^2}}$$
  $$n_z = \frac{z}{\sqrt{1 - w^2}}$$
\item Angle Axis to Quaternion:
  $$w = \cos \frac{\theta}{2}$$
  $$x = \sin \frac{\theta}{2}$$
  $$y = \sin \frac{\theta}{2}$$
  $$z = \sin \frac{\theta}{2}$$
\end{itemize}

Except for the fact that anti-podal Quaternions represent the same rotation there is no other ambiguity, i.e. $q$ and
$-q$ represent the same axis angle rotation. But this is to be expected since $(n,\theta)$ and $(-n,-\theta)$ also represent
the same rotation.

\Problem{6} Assuming Euler angles are small, it is very useful to know the approximation:
$R \approx I + [\omega]_×$ , where $[.]_x$ is the skew-symmetric operator and $\omega$ is the small rotation.
Now can you write down $\frac{\partial Rv}{\partial w}|_{\omega = 0}$ where $v \in R^3$
\Sol{6}
Given that $R \approx I + [\omega]_×$ we deduce : $$Rv \approx v + [\omega]_xv$$
$$Rv \approx v + \omega \times v$$
$$Rv \approx v - v \times \omega$$
$$Rv \approx = v - [v]_x\omega$$
Now we can differentiate easily, using the fact
$$\frac{d(a^Tx)}{dx} = a^T$$
Hence we get
$$\frac{\partial Rv}{\partial w}|_{\substack{\omega = 0}} = -[v]_x$$
\end{document}
